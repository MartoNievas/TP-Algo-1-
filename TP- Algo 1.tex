\documentclass[10pt,a4paper]{article}
\usepackage{circledsteps}
\usepackage[spanish,activeacute,es-tabla]{babel}
\usepackage[utf8]{inputenc}
\usepackage{ifthen}
\usepackage{listings}
\usepackage{dsfont}
\usepackage{subcaption}
\usepackage{amsmath}
\usepackage[strict]{changepage}
\usepackage[top=1cm,bottom=2cm,left=1cm,right=1cm]{geometry}%
\usepackage{color}%
\newcommand{\tocarEspacios}{%
	\addtolength{\leftskip}{3em}%
	\setlength{\parindent}{0em}%
}

% Especificacion de procs

\newcommand{\In}{\textsf{in }}
\newcommand{\Out}{\textsf{out }}
\newcommand{\Inout}{\textsf{inout }}

\newcommand{\encabezadoDeProc}[4]{%
	% Ponemos la palabrita problema en tt
	%  \noindent%
	{\normalfont\bfseries\ttfamily proc}%
	% Ponemos el nombre del problema
	\ %
	{\normalfont\ttfamily #2}%
	\
	% Ponemos los parametros
	(#3)%
	\ifthenelse{\equal{#4}{}}{}{%
		% Por ultimo, va el tipo del resultado
		\ : #4}
}

\newenvironment{proc}[4][res]{%
	
	% El parametro 1 (opcional) es el nombre del resultado
	% El parametro 2 es el nombre del problema
	% El parametro 3 son los parametros
	% El parametro 4 es el tipo del resultado
	% Preambulo del ambiente problema
	% Tenemos que definir los comandos requiere, asegura, modifica y aux
	\newcommand{\requiere}[2][]{%
		{\normalfont\bfseries\ttfamily requiere}%
		\ifthenelse{\equal{##1}{}}{}{\ {\normalfont\ttfamily ##1} :}\ %
		\{\ensuremath{##2}\}%
		{\normalfont\bfseries\,\par}%
	}
	\newcommand{\asegura}[2][]{%
		{\normalfont\bfseries\ttfamily asegura}%
		\ifthenelse{\equal{##1}{}}{}{\ {\normalfont\ttfamily ##1} :}\
		\{\ensuremath{##2}\}%
		{\normalfont\bfseries\,\par}%
	}
	\renewcommand{\aux}[4]{%
		{\normalfont\bfseries\ttfamily aux\ }%
		{\normalfont\ttfamily ##1}%
		\ifthenelse{\equal{##2}{}}{}{\ (##2)}\ : ##3\, = \ensuremath{##4}%
		{\normalfont\bfseries\,;\par}%
	}
	\renewcommand{\pred}[3]{%
		{\normalfont\bfseries\ttfamily pred }%
		{\normalfont\ttfamily ##1}%
		\ifthenelse{\equal{##2}{}}{}{\ (##2) }%
		\{%
		\begin{adjustwidth}{+5em}{}
			\ensuremath{##3}
		\end{adjustwidth}
		\}%
		{\normalfont\bfseries\,\par}%
	}
	
	\newcommand{\res}{#1}
	\vspace{1ex}
	\noindent
	\encabezadoDeProc{#1}{#2}{#3}{#4}
	% Abrimos la llave
	\par%
	\tocarEspacios
}
{
	% Cerramos la llave
	\vspace{1ex}
}

\newcommand{\aux}[4]{%
	{\normalfont\bfseries\ttfamily\noindent aux\ }%
	{\normalfont\ttfamily #1}%
	\ifthenelse{\equal{#2}{}}{}{\ (#2)}\ : #3\, = \ensuremath{#4}%
	{\normalfont\bfseries\,;\par}%
}

\newcommand{\pred}[3]{%
	{\normalfont\bfseries\ttfamily\noindent pred }%
	{\normalfont\ttfamily #1}%
	\ifthenelse{\equal{#2}{}}{}{\ (#2) }%
	\{%
	\begin{adjustwidth}{+2em}{}
		\ensuremath{#3}
	\end{adjustwidth}
	\}%
	{\normalfont\bfseries\,\par}%
}

% Tipos

\newcommand{\nat}{\ensuremath{\mathds{N}}}
\newcommand{\ent}{\ensuremath{\mathds{Z}}}
\newcommand{\float}{\ensuremath{\mathds{R}}}
\newcommand{\bool}{\ensuremath{\mathsf{Bool}}}
\newcommand{\cha}{\ensuremath{\mathsf{Char}}}
\newcommand{\str}{\ensuremath{\mathsf{String}}}

% Logica

\newcommand{\True}{\ensuremath{\mathrm{true}}}
\newcommand{\False}{\ensuremath{\mathrm{false}}}
\newcommand{\Then}{\ensuremath{\rightarrow}}
\newcommand{\Iff}{\ensuremath{\leftrightarrow}}
\newcommand{\implica}{\ensuremath{\longrightarrow}}
\newcommand{\IfThenElse}[3]{\ensuremath{\mathsf{if}\ #1\ \mathsf{then}\ #2\ \mathsf{else}\ #3\ \mathsf{fi}}}
\newcommand{\yLuego}{\land _L}
\newcommand{\oLuego}{\lor _L}
\newcommand{\implicaLuego}{\implica _L}

\newcommand{\cuantificador}[5]{%
	\ensuremath{(#2 #3: #4)\ (%
		\ifthenelse{\equal{#1}{unalinea}}{
			#5
		}{
			$ % exiting math mode
			\begin{adjustwidth}{+2em}{}
				$#5$%
			\end{adjustwidth}%
			$ % entering math mode
		}
		)}
}

\newcommand{\existe}[4][]{%
	\cuantificador{#1}{\exists}{#2}{#3}{#4}
}
\newcommand{\paraTodo}[4][]{%
	\cuantificador{#1}{\forall}{#2}{#3}{#4}
}

%listas

\newcommand{\TLista}[1]{\ensuremath{seq \langle #1\rangle}}
\newcommand{\lvacia}{\ensuremath{[\ ]}}
\newcommand{\lv}{\ensuremath{[\ ]}}
\newcommand{\longitud}[1]{\ensuremath{|#1|}}
\newcommand{\cons}[1]{\ensuremath{\mathsf{addFirst}}(#1)}
\newcommand{\indice}[1]{\ensuremath{\mathsf{indice}}(#1)}
\newcommand{\conc}[1]{\ensuremath{\mathsf{concat}}(#1)}
\newcommand{\cab}[1]{\ensuremath{\mathsf{head}}(#1)}
\newcommand{\cola}[1]{\ensuremath{\mathsf{tail}}(#1)}
\newcommand{\sub}[1]{\ensuremath{\mathsf{subseq}}(#1)}
\newcommand{\en}[1]{\ensuremath{\mathsf{en}}(#1)}
\newcommand{\cuenta}[2]{\mathsf{cuenta}\ensuremath{(#1, #2)}}
\newcommand{\suma}[1]{\mathsf{suma}(#1)}
\newcommand{\twodots}{\ensuremath{\mathrm{..}}}
\newcommand{\masmas}{\ensuremath{++}}
\newcommand{\matriz}[1]{\TLista{\TLista{#1}}}
\newcommand{\seqchar}{\TLista{\cha}}

\renewcommand{\lstlistingname}{Código}
\lstset{% general command to set parameter(s)
	language=Java,
	morekeywords={endif, endwhile, skip},
	basewidth={0.47em,0.40em},
	columns=fixed, fontadjust, resetmargins, xrightmargin=5pt, xleftmargin=15pt,
	flexiblecolumns=false, tabsize=4, breaklines, breakatwhitespace=false, extendedchars=true,
	numbers=left, numberstyle=\tiny, stepnumber=1, numbersep=9pt,
	frame=l, framesep=3pt,
	captionpos=b,
}

\usepackage{caratula} % Version modificada para usar las macros de algo1 de ~> https://github.com/bcardiff/dc-tex
\usepackage{amssymb}

\titulo{Trabajo Práctico de Especificación}
\subtitulo{Especificación y WP(Weakness Precondition)}

\fecha{\today}

\materia{Algoritmos y Estructuras de Datos 1}
\grupo{pesutipolimardiano}

\integrante{Nievas, Martin}{453/24}{tinnivas@gmail.com}
\integrante{Bercovich, Maximo}{504/24}{maximobercovich@gmail.com}
\integrante{Apellido, Nicolas}{003/01}{email3@dominio.com}
\integrante{Pomsztein, Andy}{624/24}{pomszteinandy@gmail.com}
\graphicspath{{../static/}}

\begin{document}

\maketitle
\section {Enunciados}
\subsection{Especificación}

\begin{quote}
	\vspace{0.2cm}
	\textbf{1. grandesCiudades:} A partir de una lista de ciudades, devuelve aquellas que tienen más de 50.000 habitantes.  
	\textbf{proc grandesCiudades} (\In ciudades: \TLista{Ciudad}) : \TLista{Ciudad}
	\vspace{0.2cm}
	\\
	\vspace{0.2cm}
	\textbf{2. sumaDeHabitantes:} Por cuestiones de planificación urbana, las ciudades registran sus habitantes mayores de edad por un lado y menores de edad por el otro. Dadas dos listas de ciudades del mismo largo con los mismos nombres, una con sus habitantes mayores y otra con sus habitantes menores, este procedimiento debe devolver una lista de ciudades con la cantidad total de sus habitantes.
	\\
	\vspace{0.2cm}
	\textbf{proc sumaDeHabitantes} (\In menoresDeCiudades: \TLista{Ciudad}, \In mayoresDeCiudades: \TLista{Ciudad}) : \TLista{Ciudad}
	\\
	\vspace{0.2cm}
	\textbf{3. hayCamino:} Un mapa de ciudades está conformado por ciudades y caminos que unen a algunas de ellas. A partir de este mapa, podemos definir las distancias entre ciudades como una matriz donde cada celda i, j representa la distancia entre la ciudad i y la ciudad j (Fig. 2). Una distancia de 0 equivale a no haber camino entre i y j. Notar que la distancia de una ciudad hacia sí misma es cero y la distancia entre A y B es la misma que entre B y A.
	\vspace{0.2cm}
	\\
	\vspace{0.2cm}
	\textbf{proc hayCamino} (\In distancias: \matriz{\ent}, \In desde: \ent, \In hasta: \ent) : \bool
	\vspace{0.2cm}
	\\
	\vspace{0.2cm}
	\textbf{4. cantidadCaminosNSaltos:} Dentro del contexto de redes informáticas, nos interesa contar la cantidad de “saltos” que realizan los paquetes de datos, donde un salto se define como pasar por un nodo.  
	Así como definimos la matriz de distancias, podemos definir la matriz de conexión entre nodos, donde cada celda i, j tiene un 1 si hay un único camino a un salto de distancia entre el nodo i y el nodo j, y un 0 en caso contrario. En este caso, se trata de una matriz de conexión de orden 1, ya que indica cuáles pares de nodos poseen 1 camino entre ellos a 1 salto de distancia.  
	Dada la matriz de conexión de orden 1, este procedimiento debe obtener aquella de orden n que indica cuántos caminos de n saltos hay entre los distintos nodos. Notar que la multiplicación de una matriz de conexión de orden 1 consigo misma nos da la matriz de conexión de orden 2, y así sucesivamente.
	\\
	\vspace{0.2cm}
	\textbf{proc cantidadNSaltos} (\Inout conexion: \matriz{\ent}, \In n: \ent)
	\vspace{0.2cm}
	\\
	\vspace{0.2cm}
	\textbf{5. caminoMínimo:} Dada una matriz de distancias, una ciudad de origen y una ciudad de destino, este procedimiento debe devolver la lista de ciudades que conforman el camino más corto entre ambas. En caso de no existir un camino, se debe devolver una lista vacía.
	\\
	\textbf{proc caminoMinimo} (\In origen: \ent, \In destino: \ent, \In distancias: seq⟨seq⟨\ent⟩⟩) : seq⟨\ent⟩
\end{quote}

\subsection{WP (Weakest Precondition)}
\begin{quote}
	La función \textbf{poblacionTotal} recibe una lista de ciudades donde al menos una de ellas es grande (es decir, supera los
	50.000 habitantes) y devuelve la cantidad total de habitantes. Dada la siguiente especificación:
	\begin{proc}{poblacionTotal}{\In ciudades : \TLista{Ciudad}}{\ent}
		\requiere {\existe[unalinea]{i}{\ent}{0 \leq i < \longitud{ciudades} \yLuego ciudades[i].habitantes > 50,000 \land \paraTodo[unalinea]{i}{\ent}{0 \leq i < \longitud{ciudades} \implicaLuego ciudades[i].habitantes \geq 0 } \land \paraTodo[unalinea]{i, j}{\ent}{0 \leq i < j \implicaLuego ciudades[i].nombre \neq ciudades[j].nombre }}}		
		\asegura{res = \sum_{i=0}^{\longitud{ciudades}-1}ciudades[i].habitantes}
	\end{proc}
	\text{Con la siguiente implementación:}
	\begin{lstlisting}[caption={},label=code:for]
		res := 0;
		i := 0;
		while (i < ciudades.length()) do
		res := res + ciudades[i].habitantes;
		i := i + 1
		endwhile
	\end{lstlisting}
	\begin{itemize}
		\item 1. Demostrar que la implementación es correcta con respecto a la especificación.
		\item 2. Demostrar que el valor devuelto es mayor a 50.000.
	\end{itemize}
\end{quote}

\section{Predicados Reutilizables}

\begin{quote}
	\pred{todosPositivos}{ s : \TLista{\ent}}{\paraTodo[unalinea]{i}{\ent}{0 \leq i < |s| \implicaLuego s[i] \geq 0}}
	\vspace{0.2cm}
	\pred{distanciasValidas}{distancias : \matriz{\ent}}{\paraTodo[unalinea]{i}{\ent}{0 \leq i < \longitud{distancias} \implicaLuego todosPositivos(distancias[i])}}
	\vspace*{0.2cm}
	\pred{diagonalEnCeros}{ s : \matriz{\ent}}{\paraTodo[unalinea]{i, j}{\ent}{0 \leq i < |s| \land 0 \leq j < s[i] \land i = j \implicaLuego s[i][j] = 0}}
	\vspace{0.2cm}
	\pred{esMatrizSimetrica}{ s : \matriz{\ent}}{\paraTodo[unalinea]{i, j}{\ent}{0 \leq i < |s| \land 0 \leq j < |s[i]| \implicaLuego s[i][j] = s[j][i]}}
	\vspace{0.2cm}
	\pred{esMatrizCuadrada}{ s : \matriz{\ent}}{\paraTodo[unalinea]{i}{\ent}{0 \leq i < |s| \implicaLuego |s[i]| = |s|}}
	\vspace*{0.2cm}
	\pred{esCamino}{ distancias : \matriz{\ent}, c : \TLista{\ent}, d : \ent, h : \ent}{(esMatrizCuadrada(distancias) \land |c| \geq 2) \yLuego  \paraTodo[unalinea]{e}{\ent}{e \in c \implicaLuego 0 \leq e < |distancias| \land (c[0] = d \land c[|c|-1] = h) \land \paraTodo[unalinea]{i}{\ent}{ 0 \leq i < |c|-1 \implicaLuego distancias[c[i]][c[i+1]] > 0 }}} 
	\vspace{0.2cm}
	\pred{ciudadesValidas}{ciudades : \TLista{Ciudad}}{\paraTodo[unalinea]{i}{\ent}{0 \leq i < \longitud{ciudades} \implica ciudades[i].habitantes \geq 0}}
\end{quote}
\section{Resolucion de Ejercicios}
\subsection{Especificacion}
\begin{quote}
	\textbf{Ejercicio 1:}
	\vspace{0.2cm}
	\begin{proc}{grandesCiudades}{\In ciudades : \TLista{Ciudad}}{\TLista{Ciudad}}
		\requiere{ciudadesDistintas(ciudades)\land ciudadesValidas(ciudades)}
		\asegura{|res| = cantidadCiudadesGrandes(ciudades)  \land sonTodasCiudadesGrandes(res) \land \paraTodo[unalinea]{c}{Ciudad}{c \in res \implicaLuego c \in ciudades}}
	\end{proc}	
	\pred{sonTodasCiudadesGrandes}{ ciudades : \TLista{Ciudad}}{
		(\paraTodo[unalinea]{i}{\ent}{0 \leq i < |ciudades| \implicaLuego ciudades[i].habitantes > 50000})   
	}
	\vspace{0.2cm}
	\aux{cantidadCiudadesGrandes}{ ciudades :\TLista{Ciudad}}{\ent}{\sum_{i = 0}^{|s|-1}{( \IfThenElse{s[i].habitantes > 50000}{1}{0})}
	}
	\vspace*{0.2cm}
	\textbf{Ejercicio 2:}
	\begin{proc}{sumaDeHabitantes}{\In menoresDeCiudades : \TLista{Ciudad}, \In mayoresDeCiudad : \TLista{Ciudad}}{\TLista{Ciudad}}
		\requiere{|menoresDeCiudades| = |mayoresDeCiudades|  \land \\ mismasCiudades(menoresDeCiudades, mayoresDeCiudades \land \\ ciudadesDistintas(menoresDeCiudades) \land ciudadesDistintas(mayoresDeCiudades) \land \\ ciudadesValidas(menoresDeCiudades)\land ciudadesValidas(mayoresDeCiudades))}
		\asegura{\longitud{res} = \longitud{menoresDeCiudades} \land esLaSuma(res, menoresDeCiudades, mayoresDeCiudades) \land \\ mismasCiudades(res, menoresDeCiudades) \land mismasCiudades(res, mayoresDeCiudades) }
	\end{proc}
	\pred{mismasCiudades}{ s: \TLista{Ciudad},  l: \TLista{Ciudad}}{\paraTodo[unalinea]{i}{\ent}{0 \leq i < |s| \implicaLuego \existe[unalinea]{j}{\ent}{0 \leq j < |l| \yLuego s[i].nombre = l[j].nombre}} }
	\vspace{0.2cm}
	\pred{esLaSuma}{ res : \TLista{Ciudad},  s: \TLista{Ciudad},  l: \TLista{Ciudad}}{\paraTodo[unalinea]{i}{\ent}{0 \leq i < |res| \implicaLuego \existe[unalinea]{j,k}{\ent}{0 \leq j < |s| \land 0 \leq k < |l| \yLuego s[j].nombre = l[k].nombre \land res[i].habitantes = s[j].habitantes + l[k].habitantes }}}
	\vspace{0.2cm}
	\pred{ciudadesDistintas}{ ciudades : \TLista{Ciudad}}{\paraTodo[unalinea]{i}{\ent}{0 \leq i < |ciudades| \implicaLuego \neg \existe[unalinea]{j}{\ent}{0 \leq j <|ciudades| \land i \neq j \yLuego ciudades[i].nombre = ciudades[j].nombre} }}
	\vspace*{0.2cm}
	\textbf{Ejercicio 3:}
	\begin{proc}{hayCamino}{\In distancias : \matriz{\ent}, \In desde : \ent, \In hasta : \ent}{\bool}
		\requiere{esMatrizCuadrada(distancias) \land diagonalEnCeros(distancias) \land esMatrizSimetrica(distancias) \land distanciasValidas(distancias) \land ( 0 \leq desde < |distancias| \land 0 \leq hasta < |distancias|)}
		\asegura{res = \True \longleftrightarrow \existe[unalinea]{c}{\TLista{\ent}}{esCamino (distancias, c, desde, hasta)}}
	\end{proc}
	\textbf{Ejercicio 4:} \\
	\begin{proc}{cantidadDeCaminosNSaltos}{\Inout conexion : \matriz{\ent}, n : \ent}{}
		\requiere{conexion = conexion_0 \land esMatrizCuadrada(conexion) \land cerosEnLaDiagonal(conexion) \land esMatrizSimetrica(conexion) \land esMatrizConCerosYUnos(conexion) }	
		\asegura{\longitud{conexion} = \longitud{conexion_0} \yLuego \paraTodo[unalinea]{i}{\ent}{0 \leq i < \longitud{conexion_0} \implicaLuego \longitud{conexion[i]} = \longitud{conexion_0} \land esMatrizDeOrdenN(conexion,conexion_0,n)}}
	\end{proc}
	\vspace{0.2cm}
	\pred{esMatrizConCerosYUnos}{conexion : \matriz{\ent}}{\paraTodo[unalinea]{i, j}{\ent}{0 \leq i < \longitud{conexion} \land 0 \leq j < \longitud{conexion[i]} \implicaLuego (conexion[i][j] = 0 \vee conexion[i][j] = 1)}}
	\vspace*{0.2cm}
	\pred{esIdentidad}{m : \matriz{\ent}}{(esMatrizCuadra) \yLuego \paraTodo[unalinea]{i, j}{\ent}{0 \leq i < \longitud{m} \land 0 \leq j < \longitud{m[i]} \implicaLuego ((i = j \land m[i][j] = 1) \vee (i \neq j \land m[i][j] = 0))}}
	\vspace*{0.2cm}
	\pred{esProducto}{m : \matriz{\ent}, n : \matriz{\ent}, o : \matriz{\ent}, n : \ent}{\paraTodo[unalinea]{i, j}{\ent}{0 \leq i < \longitud{m} \land 0 \leq j < \longitud{m[i]} \implicaLuego m[i][j] = \sum_{k = 0}^{\longitud{n} -1} n[i][k] * o[k][j]}}
	\vspace*{0.2cm}
	\pred{esMatrizDeOrdenN}{s : \matriz{\ent}, l : \matriz{\ent}}{\existe[unalinea]{lista}{\TLista{\matriz{\ent}}}{(\longitud{lista} = n + 1 \land esIdentidad(lista[0]) \land lista[1] = l \land lista[n] = s ) \land \paraTodo[unalinea]{i}{\ent}{1 \leq i \leq n \implicaLuego (esProducto(lista[i],lista[i-1],lista[1]))}}}
	\textbf{Ejericio 5:} \\
	\begin{proc}{caminoMinimo}{\In origen : \ent, \In destino : \ent, \In distancias : \matriz{\ent}}{\TLista{\ent}}
		\requiere{(diagonalEnCeros(distancias) \land esMatrizCuadrada(distancias) \land esMatrizSimetrica(distancias) \land distanciasValidas(distancias) \land (0 \leq origen < \longitud{distancias} \land 0 \leq destino < \longitud{distancias}))}
		\asegura{(esCamino(res) \yLuego \paraTodo[unalinea]{c}{\TLista{\ent}}{esCamino(c) \implicaLuego distanciaRecorrida(distancias, res) \geq distanciaRecorrida(distancias, c)}) \vee (res = \lvacia) \longleftrightarrow \neg \existe[unalinea]{c}{\TLista{\ent}}{esCamino(distancias, c, origen, destino)}}
	\end{proc}
	\aux{distanciaRecorrida}{distancias : \matriz{\ent}, c : \TLista{\ent}}{\ent}{\sum_{i = 0}^{\longitud{c}-2}distancias[c[i]][c[i+1]]}
\end{quote}
\section{WP (Weakest Precondition)}
\begin{quote}
 \textbf{Primer item:} 
 \\
 Para demostrar la correctitud de la implementación del programa con respecto a su especificación, debemos ver los siguientes puntos:
 \begin{quote}
 \begin{itemize}
 	\item 1. \textbf{Precondicion, guarda y postcondicion del ciclo}
 	\item 2. \textbf{P \implica $P_c$} $\Longleftrightarrow P \implica \text{Wp}(\text{res} := 0; \text{Wp}(i := 0, P_c))$(La precondición de la especificación implica la precondición del ciclo.)
 	\item 3. \textbf{Correctitud parcial del ciclo} 
 	\item 4. \textbf{Terminación del ciclo} 
 \end{itemize}
 \end{quote}
 \subsection{Precondicion, guarda y postcondicion del ciclo}
 \textbf{Precondicion del ciclo:} \\ 
 \vspace{0.2cm}
 \textbf{$P_c$} $\equiv \{(P \land i = 0 \land res = 0)\}$ (Donde P es la precondicion de la especificacion).\\
 \vspace{0.2cm}
 \textbf{Guarda del ciclo:}\\ 
 \vspace{0.2cm}
 \textbf{B} $\equiv \{(i < \longitud{ciudades})\}$ \\
 \vspace*{0.2cm}
 \textbf{Postcondicion del ciclo:} \\
 \vspace{0.2cm}
 \textbf{$Q_c$} $\equiv \{(res = \sum_{i=0}^{\longitud{ciudades}-1}ciudades[i].habitantes)\} $ \\ 
 \vspace{0.2cm}
 Una vez definido la precondicion, guarda y postcondicion ahora veamos si P (la precondicion de la especificacion) implica la $P_c$ (la precondicion de ciclo) eso lo vamos a hacer calculando la Wp(res := 0 ; Wp(i :=, $P_c$))
 \subsection{\textbf{$P \implica P_c$}.}
 \textbf{Cálculo de la Wp(res := 0 ; Wp(i := 0, $P_c$)):} \\[0.2cm]
 llamemos \Circled{1} a Wp(i := 0, $P_c$) y \Circled{2} Wp(res := 0, \Circled{1}) (utilizo el axioma 3) \\[0.2cm]
 \textbf{Calculamos \Circled{1}:} \\[0.2cm]
 $Wp(i := 0, P_c)$ $\equiv(def(i) \yLuego P \land 0 = 0 \land res = 0)$ $\equiv (P \land \True \land res = 0)$ $\equiv (P \land res = 0)$ \\[0.2cm]
 Una vez calculado \Circled{1} lo reemplazamos en \Circled{2} y nos queda $Wp(res := 0, P \land res = 0)$ \\[0.2cm]
 \textbf{Calculamos \Circled{2}} \\[0.2cm]
 $Wp(res := 0, P \land res = 0)$ $\equiv (def(res) \yLuego P \land 0 = 0)$ $\equiv (P \land \True)$ $\equiv P$ \\[0.2cm]
 Por lo tanto, como Wp(res := 0; Wp(i := 0, $P_c$)) $\equiv P$, podemos concluir que $P \implica P_c$. El siguiente paso en la demostración de correctitud consiste en verificar si el ciclo es parcialmente correcto. Para ello, aplicaremos el teorema del invariante, el cual nos permitirá demostrar que una propiedad invariante se mantiene a lo largo de cada iteración del ciclo, garantizando así la correctitud parcial. 
 \subsection{Correctitud parcial del ciclo mediante teorema del invariante.}
 Para demostrar la correctitud parcial del ciclo nesecitamos declarar el invariante, el cual explicado en palabras va a consistir en mantener a i en un rango adecuado para evitar la indefinicion y que res sea la suma hasta cierto i \\ [0.2cm]
 \textbf{Obs:} el invariante debe valer antes de cada iteracion y al finalizar cada iteracion.\\ [0.2cm] 
 \textbf{Sea \(I\) el invariante definido de la siguente manera:} \\ [0.1cm]
 \textbf{I} $\equiv(0 \leq i \leq \longitud{ciudades} \land res = \sum_{j=0}^{i-1}ciudades[j].habitantes)$ \\ [0.2cm] 
 Los siguientes axiomas son los que se deben corroborar para verificar que el ciclo es paracialmente correcto:
\begin{itemize}
 \item \Circled{1}. $P_c \implica I$
 \item \Circled{2}. $\{I \land B\} S \{I\}$
 \item \Circled{3}. $\{I \land \neg B\} \implica Q_c$
 \end{itemize}
 \textbf{\Circled{1}. $P_c \implica I$} (La precondición del ciclo implica el invariante) \\ [0.1cm]
 Entonces, tenemos que $(P \land res = 0 \land i = 0) \implica (0 \leq i \leq \longitud{ciudades} \land res = \sum_{j=0}^{i-1}ciudades[j].habitantes)$. Como $i = 0$, se cumple que $0 \leq i \leq \longitud{ciudades}$. Además, como consecuencia, $res = \sum_{j=0}^{0-1}ciudades[j].habitantes = \sum_{j=0}^{-1}ciudades[j].habitantes = 0$. Por lo tanto, queda demostrado que $P_c \implica I$.\\ [0.2cm] 
 \textbf{\Circled{2}. $\{I \land B\} S \{I\} \longleftrightarrow (\{I \land B\} \implica Wp(S_c, I))$} \\ [0.2cm]
 Calculemos el $Wp(S_c, I)$ (donde $S_c$ es el cuerpo del ciclo). Por lo tanto, según el axioma 3, sería equivalente calcular $Wp(res := res + ciudades[i].habitantes, Wp(i := i + 1, I))$. \\
 
 Llamemos \Circled{A} a $Wp(i := i + 1, I)$ y \Circled{B} a $Wp(res := res + ciudades[i].habitantes, \Circled{A})$. \\ [0.1cm]
 
 \textbf{Cálculo de \Circled{A}:} \\ [0.1cm]
 $Wp(i := i + 1, I) \equiv (def(i) \land (0 \leq i + 1 \leq \longitud{ciudades} \land res = \sum_{j=0}^{(i+1)-1}ciudades[j].habitantes)) \equiv (0 \leq i + 1 \leq \longitud{ciudades} \land res = \sum_{j=0}^{i}ciudades[j].habitantes)$. \\ [0.1cm]
 
 Una vez calculado \Circled{A}, reemplazamos en \Circled{B}, que queda de la siguiente manera: $Wp(res := res + ciudades[i].habitantes, 0 \leq i + 1 \leq \longitud{ciudades} \land res = \sum_{j=0}^{i}ciudades[j].habitantes)$. \\ [0.1cm]
 
 \textbf{Cálculo de \Circled{B}:} \\ [0.1cm]
 $Wp(res := res + ciudades[i].habitantes, 0 \leq i + 1 \leq \longitud{ciudades} \land res = \sum_{j=0}^{i}ciudades[j].habitantes) \equiv ((def(i) \land def(ciudades) \land 0 \leq i < \longitud{ciudades}) \land (0 \leq i + 1 \leq \longitud{ciudades} \land res + ciudades[i].habitantes = \sum_{j=0}^{i}ciudades[j].habitantes))$. 
 
 Ahora, podemos comprimir el rango y restar $ciudades[i].habitantes$ de la sumatoria, que sería el último término. Entonces, obtenemos: \\ [0.1cm]
 
 $Wp(res := res + ciudades[i].habitantes, \Circled{A}) \equiv (0 \leq i < \longitud{ciudades} \land res = \sum_{j=0}^{i-1}ciudades[j].habitantes)$. \\ [0.1cm]
 
 Ahora, veamos si $\{I \land B\} \implica Wp(S_c, I)$. Donde $\{I \land B\} \equiv (0 \leq i \leq \longitud{ciudades} \land res = \sum_{j=0}^{i -1}ciudades[j].habitantes \land i < \longitud{ciudades})$. Comprimimos el rango de $i$ y obtenemos $(0 \leq i < \longitud{ciudades} \land res = \sum_{j=0}^{i -1}ciudades[j].habitantes)$. Como $\{I \land B\} \equiv Wp(S_c, I)$, entonces la implicación es válida y la tripla de Hoare es correcta. \\ [0.2cm]
 
 \textbf{\Circled{3}. $\{I \land \neg B\} \implica Q_c$} \\ [0.2cm]
 primero que nada definamos $\{I \land \neg B\}$ \\ 
 $\{I \land \neg B\} \equiv (0 \leq i \leq \longitud{ciudades} \land res = \sum_{j=0}^{i-1}ciudades[j].habitantes \land i \geq \longitud{ciudades})$ (como i es mayor o igual que la longitud de ciudades y menor o igual a la vez me queda lo siguente) \\ [0.1cm]
 $\equiv ( i = \longitud{ciudades} \land res = \sum_{j=0}^{\longitud{ciudades}-1}ciudades[j].habitantes) \equiv Q_c$ \\ [0.1cm]
Por lo tanto queda desmotrado que $\{I \land \neg B\} \implica Q_c$ y como consecuencia el ciclo resulta parcialmente correcto. \\
Ahora lo siguiente es verificar que el ciclo termina mediante el teorema de la terminacion. 
\subsection{Terminación del ciclo (mediante el teorema de la terminación).}

Para verificar que el ciclo concluye, debemos verificar los siguientes axiomas del teorema:
\begin{itemize}
	\item \textbf{\Circled{4}. $\{I \land B \land f_v = v_0\} S \{f_v < v_0\}$}
	\item \textbf{\Circled{5}. $\{I \land f_v \leq 0\} \implica \neg B$} 
\end{itemize}

Aclaraciones: $f_v$ es la función variante, la cual defino como $f_v = (\longitud{ciudades} - i)$. Dado que $f_v$ debe ser una función decreciente y la longitud de ciudades es constante mientras que $i$ crece con cada iteración, se trata de una función decreciente.

Para ejemplificar, supongamos que la lista de ciudades es ciudades = [("Buenos Aires",60000), ("Santa Rosa", 15000), ("Rosario", 25000), ("Jujuy", 20000)]. La siguiente tabla muestra una visualización de la evolución de los valores en cada iteración:

\begin{table}[h!]
	\centering
	\begin{tabular}{||l c c r||} 
		\hline
		Iteración & i & res & $f_v$ \\ [0.5ex] 
		\hline\hline
		0 & 0 & 0 & 4 \\ 
		1 & 1 & 60000 & 3 \\
		2 & 2 & 75000 & 2 \\
		3 & 3 & 100000 & 1 \\
		4 & 4 & 120000 & 0 \\
		\hline
	\end{tabular}
	\caption{Visualización de la evolución de los valores en cada iteración.}
	\label{tab:ejemplo}
\end{table}

Cuando $i = 4$, la guarda $B \equiv (i < \longitud{ciudades})$ deja de cumplirse, y se sale del ciclo, lo que confirma que $f_v \leq 0$.

\textbf{\Circled{4}. $\{I \land B \land f_v = v_0\} S \{f_v < v_0\}$}

Primero, definamos $\{I \land B \land f_v = v_0\}$:
\[
\{I \land B \land f_v = v_0\} \equiv (0 \leq i \leq \longitud{ciudades} \land res = \sum_{j=0}^{i-1}ciudades[j].habitantes \land i < \longitud{ciudades} \land v_0 = \longitud{ciudades} - i)
\]
\[
\equiv (0 \leq i < \longitud{ciudades} \land res = \sum_{j=0}^{i-1}ciudades[j].habitantes \land v_0 = \longitud{ciudades} - i)
\]
{(Se ha comprimido el rango de $i$).}

Para probar que la tripla de Hoare es válida: \\
(Se utiliza el axioma 3 de WP).
\[
\{I \land B \land f_v = v_0\} \implica Wp(res := res + ciudades[i].habitantes, Wp(i := i + 1, \longitud{ciudades} - i < v_0))
\]

Llamemos \Circled{1} a $Wp(i := i + 1, \longitud{ciudades} - i < v_0)$ y \Circled{2} a $Wp(res := res + ciudades[i].habitantes, \Circled{1})$.

\textbf{Calculamos \Circled{1}:}
\[
Wp(i := i + 1, \longitud{ciudades} - i < v_0) \equiv (def(i) \yLuego \longitud{ciudades} - (i + 1) < v_0) \equiv (\longitud{ciudades} - i - 1 < v_0)
\]

Ahora, reemplazamos el valor calculado de \Circled{1} en \Circled{2} y obtenemos:
\[
Wp(res := res + ciudades[i].habitantes, \longitud{ciudades} - i - 1 < v_0)
\]

\textbf{Calculamos \Circled{2}:} \\ [0.2cm]
$
Wp(res := res + ciudades[i].habitantes, \longitud{ciudades} - i - 1 < v_0) \equiv (def(i) \land def(res) \land def(ciudades) \land 0 \leq i < \longitud{ciudades} \yLuego \longitud{ciudades} - i - 1 < v_0) \equiv (0 \leq i < \longitud{ciudades} \land \longitud{ciudades} - i - 1 < v_0)
$

Luego, verificamos si $\{I \land B \land f_v = v_0\} \implica Wp(S_c, f_v < v_0)$. Esto es claramente cierto, ya que $v_0 = \longitud{ciudades} - i$ y sabemos que $v_0 > \longitud{ciudades} - i - 1$, lo que implica que $\longitud{ciudades} - i > \longitud{ciudades} - i - 1$, y por lo tanto la desigualdad $(0 > -1)$ siempre es verdadera.

\textbf{\Circled{5}. $\{I \land f_v \leq 0\} \implica \neg B$}

Primero definamos $\{I \land f_v \leq 0\}$: \\ [0.1cm]
$
\{I \land f_v \leq 0\} \equiv (0 \leq i \leq \longitud{ciudades} \land res = \sum_{j=0}^{i-1}ciudades[j].habitantes \land \longitud{ciudades} \leq i) \equiv (i = \longitud{ciudades} \land res = \sum_{j=0}^{\longitud{ciudades}-1}ciudades[j].habitantes)
$

Desarrollando la negación de la guarda:
\[
\neg B \equiv (i \geq \longitud{ciudades})
\]
Entonces, $(i = \longitud{ciudades}) \implica (i \geq \longitud{ciudades})$.

En conclusión, la correctitud de la implementación del programa ha sido demostrada mediante todo lo detallado anteriormente. Además, se ha comprobado que el ciclo es parcialmente correcto utilizando el teorema del invariante y que termina en una cantidad finita de pasos aplicando el teorema de la terminación. \\ [0.2cm]
\textbf{Segundo ítem:} \\ [0.2cm]
Como segundo punto, debemos demostrar que la suma de todos los habitantes es mayor a 50,000. Definimos $P$ como la precondición requerida en la especificación y $Q_0$ como la nueva postcondición: \\ [0.2cm]

$P \equiv \existe[unalinea]{i}{\ent}{0 \leq i < \longitud{ciudades} \yLuego ciudades[i].habitantes > 50,000 \land \paraTodo[unalinea]{i}{\ent}{0 \leq i < \longitud{ciudades} \implicaLuego ciudades[i].habitantes \geq 0 } \land \paraTodo[unalinea]{i, j}{\ent}{0 \leq i < j \implicaLuego ciudades[i].nombre \neq ciudades[j].nombre }} $ \\ [0.2cm]
$Q_0 \equiv (res = \sum_{i=0}^{\longitud{ciudades}-1}ciudades[i].habitantes \land res > 50,000)$ \\ [0.2cm]

Ahora, lo que tenemos que demostrar es que la tripla de Hoare es válida $\{P\} S \{Q_0\}$. Para eso, haremos la demostración inversa siguiendo los pasos indicados a continuación, utilizando lo demostrado en el punto anterior: 
\begin{itemize}
	\item \textbf{\Circled{1} $Q_c \implica Q_0$}
	\item \textbf{\Circled{2} $\{I \land \neg B\} \implica Q_c$}
	\item \textbf{\Circled{3} $\{I \land B\} S \{I\}$}
	\item \textbf{\Circled{4} $P_c \implica I$}
	\item \textbf{\Circled{5} $P \implica P_c$}
\end{itemize}

Comenzamos con \textbf{\Circled{1}}. Definimos $Q_c$ de la siguiente manera: \\ [0.1cm]
$Q_c \equiv (res = \sum_{i=0}^{\longitud{ciudades}-1}ciudades[i].habitantes \land res > 50,000)$ \\ [0.2cm]
Es trivial que $Q_c \implica Q_0$, ya que ambas expresiones son equivalentes. \\ [0.2cm]

A continuación, debemos probar \textbf{\Circled{2}}. Como modificamos $Q_c$, también debemos modificar el invariante para asegurarnos de que sigue cumpliéndose. Definimos entonces el nuevo invariante $(I_n)$ de la siguiente manera: \\ [0.2cm]
$I_n \equiv (0 \leq i \leq \longitud{ciudades} \yLuego res = \sum_{j=0}^{i-1}ciudades[j].habitantes \land \existe[unalinea]{k}{\ent}{0 \leq k < \longitud{ciudades} \yLuego ciudades[k].habitantes > 50,000} \land \paraTodo[unalinea]{k}{\ent}{0 \leq k < \longitud{ciudades} \implicaLuego ciudades[k].habitantes \geq 0 })$ \\ [0.2cm]

Ahora, con el nuevo invariante, la demostración es casi trivial. Sabemos que $\neg B \equiv (i \geq \longitud{ciudades})$ sigue siendo la misma. Dado el rango de $i$ en $I_n$ y el rango de $i$ en $\neg B$, podemos determinar que $i = \longitud{ciudades}$. Luego, la sumatoria queda como $\sum_{j=0}^{\longitud{ciudades}-1} ciudades[j].habitantes$, y como el invariante establece que existe al menos una ciudad con más de 50,000 habitantes, y el resto de las ciudades tiene 0 o más habitantes, se cumple $Q_c$. \\ [0.2cm]

El agregado al invariante $(I_n)$ no modifica la demostración de \textbf{\Circled{3}}, por lo que sigue siendo válida. \\ [0.2cm]

En cuanto al punto \textbf{\Circled{4}}, recordemos que $P_c$ estaba definido de la siguiente manera: \\ [0.2cm]
\textbf{$P_c$} $\equiv (P \land i = 0 \land res = 0)$ (donde $P$ es la precondición de la especificación). \\ [0.2cm]
Lo que tenía el invariante anterior sigue siendo válido, como se demostró en el punto 2.1. Ahora debemos verificar si $P_c$ implica lo que fue modificado. \\ [0.2cm]
Como $P$ forma parte de $P_c$, es fácil ver que: \\ [0.1cm]
$P_c \implica \existe[unalinea]{k}{\ent}{0 \leq k < \longitud{ciudades} \land ciudades[k].habitantes > 50,000}$ \\ 
$P_c \implica \paraTodo[unalinea]{k}{\ent}{0 \leq k < \longitud{ciudades} \implicaLuego ciudades[k].habitantes \geq 0}$ \\ [0.2cm]

Por último, en cuanto al punto \textbf{\Circled{5}}, dado que ni $P$ ni $P_c$ se modificaron, sigue siendo válido, como se demostró en el ejercicio 2.1. \\ [0.2cm]
\begin{itemize}
	\item \textbf{\Circled{1} $Q_c \implica Q_0$} \checkmark
	\item \textbf{\Circled{2} $\{I \land \neg B\} \implica Q_c$} \checkmark
	\item \textbf{\Circled{3} $\{I \land B\} S \{I\}$} \checkmark
	\item \textbf{\Circled{4} $P_c \implica I$} \checkmark
	\item \textbf{\Circled{5} $P \implica P_c$} \checkmark
\end{itemize}
\textbf{Observación:} \\[0.2cm]
Lo modificado en el invariante no perjudica la demostración de la terminación del ciclo, pero revisemos. Recordemos que $f_v = \longitud{ciudades} - i$. \\[0.2cm]
Veamos $\{I_n \land B \land f_v = v_0\} S \{f_v < v_0\}$. Como ya tengo calculada la $Wp(S_c, f_v < v_0)$, compruebo si sigue valiendo:

\[
\{I_n \land B \land f_v = v_0\} \implies Wp(S_c, f_v < v_0)
\] \\[0.2cm]
Entonces, la $Wp(S_c, f_v < v_0)$ es:

\[
(0 \leq i < \longitud{ciudades} \land \longitud{ciudades} - i - 1 < v_0)
\]

Ahora, si recordamos lo agregado al nuevo invariante $(I_n)$, fue:

\begin{itemize}
	\item $\existe[unalinea]{k}{\ent}{0 \leq k < \longitud{ciudades} \land ciudades[k].habitantes > 50,000}$
	\item $\paraTodo[unalinea]{k}{\ent}{0 \leq k < \longitud{ciudades} \implica ciudades[k].habitantes \geq 0}$
\end{itemize}

Esto no modifica la demostración, ya que $v_0 = \longitud{ciudades} - i \implies \longitud{ciudades} - 1 - i < \longitud{ciudades} - i$, lo cual nos deja con la expresión $-1 < 0$. \\[0.2cm]

Entonces, ahora veamos si sigue valiendo $\{I_n \land \longitud{ciudades} - i \leq 0\} \implies \neg B$. Si desarrollamos $\neg B \equiv (i \geq \longitud{ciudades})$ y consideramos lo agregado al invariante (lo mismo que se mencionó antes), no se modifica el rango de $i$, lo que nos deja en el mismo punto que en el paso 2.1, donde $i = \longitud{ciudades} \implies i \geq \longitud{ciudades}$. \\[0.2cm]

Por lo tanto, el ciclo sigue finalizando en pasos finitos y cumple que: 

\begin{itemize}
	\item \textbf{\Circled{6}. $\{I \land B \land f_v = v_0\} S \{f_v < v_0\}$} \checkmark
	\item \textbf{\Circled{7}. $\{I \land f_v \leq 0\} \implies \neg B$} \checkmark
\end{itemize}

Con todo esto, podemos afirmar que la tripla de Hoare $\{P\} S \{Q_0\}$ es válida, y queda demostrado que la suma de todos los habitantes es siempre mayor a 50,000.
\end{quote}
\end{document}
